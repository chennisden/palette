% \iffalse meta-comment
% This package was written for pdfLaTeX.
% In particular, there is no guarantee that this works for other tex engines.
%
% Copyright (C) 2021 Dennis Chen <proofprogram@gmail.com>
%
% This work may be distributed and/or modified under
% the conditions the LaTeX Project Public License (LPPL),
% either version 1.3 of this license or (at your option)
% any later version. The latest version of this license
% can be found in
%   http://www.latex-project.org/lppl.txt
% and version 1.3 or later is part of all distributions of LaTeX
% version 2005/12/01 or later.
%
%<*internal>
\iffalse
%</internal>
%<*readme>
# The LaTeX package palette - version 1.0.0 (2021/07/27)

**palette** provides the ability to create custom color palettes and symbol palettes ("symbol suites"), and swap one palette in for another easily. It also allows the user to create a variable that represents the selected palette, and provides checks in case the variable represents a palette that does not exist.

The package is split into `colorpalette` and `symbolsuite`.

## Installation instructions

A working TeX installation is required.

This is a self-extracting dtx file, so run

    pdflatex palette.dtx

to generate the README, documentation, and packages.
%</readme>
% \fi

% \iffalse

%<*internal>
\fi
\def\nameofplainTeX{plain}
\ifx\fmtname\nameofplainTeX\else
    \expandafter\begingroup
\fi
%</internal>
%<*install>

\input docstrip.tex
\keepsilent
\askforoverwritefalse
\preamble

This is a generated file.

Copyright (C) 2021 Dennis Chen <proofprogram@gmail.com>

This work may be distributed and/or modified under
the conditions the LaTeX Project Public License (LPPL),
either version 1.3 of this license or (at your option)
any later version. The latest version of this license
can be found in

    http://www.latex-project.org/lppl.txt

and version 1.3 or later is part of all distributions of LaTeX
version 2005/12/01 or later.

To produce the documentation, run palette.dtx through pdflatex.

\endpreamble
\postamble

This package consists of the file palette.dtx,
          and the generated files colorpalette.sty,
                                  symbolpalette.sty,
                                  palette.pdf.
\endpostamble

\usedir{tex/latex/palette}
\generate{
    \file{colorpalette.sty}{\from{\jobname.dtx}{color}}
    \file{symbolpalette.sty}{\from{\jobname.dtx}{symbol}}
}

\obeyspaces
\Msg{*************************************************************}
\Msg{*                                                           *}
\Msg{* To finish the installation you have to move the following *}
\Msg{* files into a directory searched by TeX:                   *}
\Msg{*                                                           *}
\Msg{*     colorpalette.sty                                      *}
\Msg{*     symbolpalette.sty                                       *}
\Msg{*                                                           *}
\Msg{* To produce the documentation, run the file                *}
\Msg{* palette.dtx through LaTeX.                                *}
\Msg{*                                                           *}
\Msg{* Happy TeXing!                                             *}
\Msg{*                                                           *}
\Msg{*************************************************************}

%</install>
%<install>\endbatchfile
%<*internal>

\usedir{source/latex/palette}
\generate{
    \file{\jobname.ins}{\from{\jobname.dtx}{install}}
}
\nopreamble\nopostamble

\usedir{doc/latex/palette}
\generate{
    \file{README.md}{\from{\jobname.dtx}{readme}}
}

\ifx\fmtname\nameofplainTeX
    \expandafter\endbatchfile
\else
    \expandafter\endgroup
\fi
%</internal>
%
%<color|symbol>\NeedsTeXFormat{LaTeX2e}
%<color>\ProvidesPackage{colorpalette}
%<symbol>\ProvidesPackage{symbolsuite}
%<color|symbol>[2021/07/27 v1.0.0 Create palettes for colors and symbols]
%<color>\RequirePackage{xcolor}
%<*color|symbol>
\RequirePackage{macrolist}
%</color|symbol>
%<*driver>
\ProvidesFile{\jobname.dtx}[2021/07/27 v1.0.0 Create palettes for colors and symbols]
\documentclass{ltxdoc}

\EnableCrossrefs
\CodelineIndex
\RecordChanges

\begin{document}
    \DocInput{\jobname.dtx}
    \PrintIndex
    \PrintChanges
\end{document}
%</driver>
% \fi
%
% \GetFileInfo{\jobname.dtx}
%
% \changes{v1.0.0}{2021/07/27}{Initial version}
%
% \title{\textsf{palette} -- Create palettes for colors and symbols}
% \author{Dennis Chen \\ proofprogram@gmail.com}
% \date{\fileversion{} (\filedate)}
%
% \maketitle
%
% \begin{abstract}
%   The \textsf{palette} package provides \textsf{colorpalette.sty} and \textsf{symbolpalette.sty}, which can be used to create and manage palettes of colors and symbols, respectively.
% \end{abstract}
%
% \section{Overview}
% We will go over the basic structure of the user interface in broad terms first (i.e. no macros or environments yet). We will use \textsf{colorpalette} as our example here, but \textsf{symbolsuite} behaves virtually identically.
%
% First, the user can create palette \emph{themes}, which are independent from each other. Then, you can add colors to themes. Finally, you can make a palette associated with a theme and set its colors.
%
% \section{colorpalette.sty}
%
% \subsection{Usage}
%
% \DescribeMacro{\newpalettetheme}
% To create a new palette theme called \textsf{theme}, write |\newpalettetheme{theme}|.
%
% \iffalse
%<*color>
\newcommand\newpalettetheme[1]{
    \newlist{colorpalette@theme@#1@palettes}
    \newlist{colorpalette@theme@#1@colors}
}
%</color>
% \fi
% \DescribeMacro{\addcolortotheme}
% To add a color \textsf{color} to theme \textsf{theme}, write |\addcolortotheme{theme}{color}|.
%
% \iffalse
%<*color>
\newcommand\addcolortotheme[2]{
    \colorpalette@themecheck{#1}
    \listadd{colorpalette@theme@#1@colors}{#2}
}
%</color>
% \fi
%
% \DescribeMacro{\newpalette}
% To create a new palette \textsf{palette} based on theme \textsf{theme}, write |\newpalette{theme}{palette}|.
%
% Palette names only need to be unique across the same theme. This means that you can have two palettes with the same name if they are associated with different themes.
%
% \iffalse
%<*color>
\newcommand\newpalette[2]{
    % Check that the theme exists
    \colorpalette@themecheck{#1}
    \listadd{colorpalette@theme@#1@palettes}{#2}
}
%</color>
% \fi
%
% \DescribeMacro{\setpalettecolor}
% To set the color \textsf{color} (which was defined with respect to the theme of the palette) as the color value \textsf{value} in the format \textsf{encoding} (which by default is HTML) of palette \textsf{palette} in theme \textsf{theme}, write \begin{verbatim}\setpalettecolor{theme}{palette}{color}[encoding]{value}\end{verbatim}
%
% \iffalse
%<*color>
\newcommand\setpalettecolor[3]{
    \colorpalette@themecheck{#1}
    \colorpalette@palettecheck{#1}{#2}
    \colorpalette@colorcheck{#1}{#3}
    \def\colorpalette@temp@themename{#1}
    \def\colorpalette@temp@palettename{#2}
    \def\colorpalette@temp@colorname{#3}
    \colorpalette@setpalettecolor
}
\newcommand\colorpalette@setpalettecolor[2][HTML]{
    \definecolor{colorpalette@\colorpalette@temp@themename @\colorpalette@temp@palettename @\colorpalette@temp@colorname}{#1}{#2}
}
%</color>
% \fi
%
% \DescribeMacro{\activepalette}
% This sets the active palette of a theme. (This is how palettes are ``swapped in'' with respect to a theme.) To set the active palette of theme \textsf{theme} as \textsf{palette}, write |\activepalette{theme}{palette}|.
%
% \iffalse
%<*color>
\newcommand\activepalette[2]{%
    \colorpalette@themecheck{#1}
    \colorpalette@palettecheck{#1}{#2}
    \expandafter\def\csname colorpalette@#1@active\endcsname{#2}%
}
%</color>
% \fi
%
% \DescribeMacro{\getcolor}
% To use, write |\getcolor{theme}{color}|. This gets the \emph{internal} color name from the \emph{active} palette of \textsf{theme}. No errors will be thrown even if you pass in an undefined theme or color (with respect to the theme). This limitation is because we must be able to parse |\getcolor|.
%
% \iffalse
%<*color>
\newcommand\getcolor[2]{colorpalette@#1@\csname colorpalette@#1@active\endcsname @#2}
%</color>
% \fi
%
% \DescribeMacro{\applycolor}
% To use, write |\applycolor{theme}{color}|. This applies color \textsf{color} from the \emph{active} palette of theme \textsf{theme}. (This means that the newly defined color is passed into |\color|.) If \textsf{theme} has no active palette, an error will be thrown.
%
% To make it clearer what |\applycolor| does, we will just show the macro definition directly (since it is quite simple). (The macros |\colorpalette@themecheck| and |\colorpalette@colorcheck| are just for error handling.)
%
%    \begin{macrocode}
%<*color>
\newcommand\applycolor[2]{%
    \colorpalette@themecheck{#1}%
    \colorpalette@colorcheck{#1}{#2}%
    \color{\getcolor{#1}{#2}}%
}
%</color>
%    \end{macrocode}
%
% \subsection{Error Messages}
% \iffalse
%<*color>
\newcommand\colorpalette@themecheck[1]{\listexists{colorpalette@theme@#1@palettes}{}{\colorpalette@error@theme{#1}}}
\newcommand\colorpalette@palettecheck[2]{\listcontains{colorpalette@theme@#1@palettes}{#2}{}{\colorpalette@error@palette{#1}{#2}}}
\newcommand\colorpalette@colorcheck[2]{\listcontains{colorpalette@theme@#1@colors}{#2}{}{\colorpalette@error@color{#1}{#2}}}
%</color>
% \fi
%
% If a nonexistent theme is passed in, the following error will be thrown:
% \begin{verbatim}The palette passed in is not a defined palette of theme `#1'\end{verbatim}
% where |#1| is the name of the theme.
% \iffalse
%<color>\newcommand\colorpalette@error@theme[1]{\PackageError{colorpalette}{The theme `#1' passed in is not defined}{}}
% \fi
%
% If a palette that does not belong to a theme is passed in, the following error will be thrown:
% \begin{verbatim}The palette `#2' passed in is not a defined palette of theme `#1'\end{verbatim}
% where |#1| is the name of the theme and |#2| is the name of the palette.
% \iffalse
%<color>\newcommand\colorpalette@error@palette[2]{\PackageError{colorpalette}{The palette `#2' passed in is not a defined palette of theme `#1'}{}}
% \fi
%
% If a color that does not belong to a theme is passed in, the following error will be thrown:
% \begin{verbatim}The color `#2' passed in is not a defined color of theme `#1'\end{verbatim}
% where |#1| is the name of the theme and |#2| is the name of the color.
% \iffalse
%<color>\newcommand\colorpalette@error@color[2]{\PackageError{colorpalette}{The color `#2' passed in is not a defined color of theme `#1'}{}}
% \fi
%
% \section{symbolpalette.sty}
%
% Despite the name \textsf{symbolpalette} (which is used to identify it as one of the files of the package \textsf{palette}), the main keyword is a \emph{suite}. This is because the keyword \emph{palette} is already used for \textsf{colorpalette}.
%
% This package is nearly analogous to \textsf{colorpalette}. A \emph{suite} is analogous to a \emph{palette}, and a \emph{symbol} is analogous to a \emph{color}. \emph{Themes} are used in both packages.
%
% \subsection{Usage}
%
% \DescribeMacro{\newsuitetheme}
% To create a new suite theme called \textsf{theme}, write |\newsuitetheme{theme}|.
%
% \iffalse
%<*symbol>
\newcommand\newsuitetheme[1]{
    \newlist{symbolsuite@theme@#1@suites}
    \newlist{symbolsuite@theme@#1@symbols}
}
%</symbol>
% \fi
% \DescribeMacro{\addsymboltotheme}
% To add a symbol \textsf{symbol} to theme \textsf{theme}, write |\addsymboltotheme{theme}{symbol}|.
%
% \iffalse
%<*symbol>
\newcommand\addsymboltotheme[2]{
    \symbolsuite@themecheck{#1}
    \listadd{symbolsuite@theme@#1@symbols}{#2}
}
%</symbol>
% \fi
%
% \DescribeMacro{\newsuite}
% To create a new suite \textsf{suite} based on theme \textsf{theme}, write |\newsuite{theme}{suite}|.
%
% Suite names only need to be unique across the same theme. This means that you can have two suites with the same name if they are associated with different themes.
%
% \iffalse
%<*symbol>
\newcommand\newsuite[2]{
    \symbolsuite@themecheck{#1}
    \listadd{symbolsuite@theme@#1@suites}{#2}
}
%</symbol>
% \fi
%
% \DescribeMacro{\setsuitesymbol}
% To set the symbol \textsf{symbol} (which was defined with respect to the theme of the suite) as the icon \textsf{icon} of suite \textsf{suite} in theme \textsf{theme}, write \begin{verbatim}\setsuitesymbol{theme}{suite}{symbol}{icon}\end{verbatim}
%
% \iffalse
%<symbol>\newcommand\setsuitesymbol[4]{\expandafter\def\csname symbolsuite@#1@#2@#3\endcsname{#4}}
% \fi
%
% \DescribeMacro{\activesuite}
% This sets the active suite of a theme. (This is how suites are ``swapped in'' with respect to a theme.) To set the active suite of theme \textsf{theme} as \textsf{suite}, write |\activesuite{theme}{suite}|.
%
% \iffalse
%<*symbol>
\newcommand\activesuite[2]{
    \symbolsuite@themecheck{#1}
    \symbolsuite@suitecheck{#1}{#2}
    \expandafter\def\csname symbolsuite@#1@active\endcsname{#2}
}
%</symbol>
% \fi
%
% \DescribeMacro{\printsymbol}
% To use, write |\printsymbol{theme}{symbol}|. This is similar to \textsf{colorpalette.sty}'s |\applycolor|.
% \iffalse
%<*symbol>
\newcommand\printsymbol[2]{%
    \symbolsuite@themecheck{#1}%
    \symbolsuite@symbolcheck{#1}{#2}%
    \csname symbolsuite@#1@\csname symbolsuite@#1@active\endcsname @#2\endcsname%
}
%</symbol>
% \fi
%
% \subsection{Error Messages}
% \iffalse
%<*symbol>
\newcommand\symbolsuite@themecheck[1]{\listexists{symbolsuite@theme@#1@suites}{}{\symbolsuite@error@theme{#1}}}
\newcommand\symbolsuite@suitecheck[2]{\listcontains{symbolsuite@theme@#1@suites}{#2}{}{\symbolsuite@error@suite{#1}{#2}}}
\newcommand\symbolsuite@symbolcheck[2]{\listcontains{symbolsuite@theme@#1@symbols}{#2}{}{\symbolsuite@error@symbol{#1}{#2}}}
%</symbol>
% \fi
%
% If a nonexistent theme is passed in, the following error will be thrown:
% \begin{verbatim}The palette passed in is not a defined palette of theme `#1'\end{verbatim}
% where |#1| is the name of the theme.
% \iffalse
%<symbol>\newcommand\symbolsuite@error@theme[1]{\PackageError{symbolsuite}{The theme `#1' passed in is not defined}{}}
% \fi
%
% If a suite that does not belong to a theme is passed in, the following error will be thrown:
% \begin{verbatim}The suite `#2' passed in is not a defined suite of theme `#1'\end{verbatim}
% where |#1| is the name of the theme and |#2| is the name of the suite.
% \iffalse
%<symbol>\newcommand\symbolsuite@error@suite[2]{\PackageError{symbolsuite}{The suite `#2' passed in is not a defined suite of theme `#1'}{}}
% \fi
%
% If a symbol that does not belong to a theme is passed in, the following error will be thrown:
% \begin{verbatim}The symbol `#2' passed in is not a defined symbol of theme `#1'\end{verbatim}
% where |#1| is the name of the theme and |#2| is the name of the symbol.
% \iffalse
%<symbol>\newcommand\symbolsuite@error@symbol[2]{\PackageError{symbolsuite}{The symbol `#2' passed in is not a defined color of theme `#1'}{}}
% \fi
% \Finale
%
\endinput